\plainsection{SỐ PHÁCH VÀ KẾT QUẢ CHẤM}
%\addcontentsline{toc}{section}{SỐ PHÁCH VÀ KẾT QUẢ CHẤM}

\vspace{0.3cm} % Thêm khoảng cách dưới tiêu đề

% Bảng "Số phách" căn phải
\begin{flushright} % Căn bảng về bên phải trang
\renewcommand{\arraystretch}{1.5} % Tăng khoảng cách giữa các dòng trong bảng
\fontsize{14pt}{18pt}\selectfont % Thiết lập cỡ chữ 14pt, khoảng cách dòng 18pt
\begin{tabular}{|p{5cm}|} % Bảng gồm 1 cột, độ rộng 5cm, có viền
  \hline
  \centering \textbf{Số phách} \tabularnewline % Ô đầu tiên: in đậm và căn giữa chữ "Số phách"
  \hline
  \rule{0pt}{2cm} \tabularnewline % Ô trống chiều cao 2cm (chứa số phách viết tay)
  \hline
\end{tabular}
\end{flushright}

\vspace{0.5cm} % Khoảng cách giữa 2 bảng

% Bảng "Kết quả chấm"
\renewcommand{\arraystretch}{1.5} % Tăng khoảng cách dòng trong bảng
\fontsize{14pt}{18pt}\selectfont % Thiết lập cỡ chữ 14pt, khoảng cách dòng 18pt
\begin{flushright} % Căn bảng về bên phải
\begin{tabular}{|>{\centering\arraybackslash}m{6.5cm}|>{\centering\arraybackslash}m{4cm}|>{\centering\arraybackslash}m{4cm}|}
% Bảng có 3 cột: 
% Cột 1 rộng 6.5cm: chứa thông tin cán bộ chấm thi hoặc điểm kết luận
% Cột 2 và 3 rộng 4cm: chứa điểm bằng số và chữ, tất cả đều căn giữa
\hline
\rule{0pt}{1.2cm} & \textbf{Điểm bằng số} & \textbf{Điểm bằng chữ} \tabularnewline % Hàng tiêu đề bảng
\hline

% Cán bộ chấm thi thứ nhất

\begin{minipage}[t][4.5cm][t]{\linewidth} % Ô đầu tiên dùng minipage: cao 4.5cm, canh đầu ô theo chiều dọc
    \centering % Căn giữa theo chiều ngang
    \textbf{Cán bộ chấm thi thứ nhất} \\[6pt] % Tiêu đề in đậm, xuống dòng cách 6pt
    (Ký, ghi rõ họ tên) % Ghi chú dưới
\end{minipage}

& & \tabularnewline % Hai ô còn lại để trống
\hline

% Cán bộ chấm thi thứ hai

\begin{minipage}[t][4.5cm][t]{\linewidth} % Thiết lập tương tự như cán bộ thứ nhất
    \centering
    \textbf{Cán bộ chấm thi thứ hai} \\[6pt]
    (Ký, ghi rõ họ tên)
\end{minipage}

& & \tabularnewline
\hline

% Điểm kết luận

\begin{minipage}[t][2cm][c]{\linewidth} % Dòng này dùng minipage cao 2.5cm, canh giữa theo chiều dọc
    \centering
    \textbf{Điểm kết luận} % In đậm, căn giữa
\end{minipage}

& & \tabularnewline
\hline
\end{tabular}
\end{flushright}
