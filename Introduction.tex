\plainsection{MỞ ĐẦU}
\addcontentsline{toc}{section}{MỞ ĐẦU}

\textbf{1. Lý do chọn đề tài}

Trong thời đại số, công nghệ điện toán đám mây đã trở thành nền tảng quan trọng giúp các tổ chức tối ưu hóa nguồn lực, giảm chi phí đầu tư hạ tầng và nâng cao năng lực cạnh tranh. Thay vì phải duy trì hệ thống máy chủ vật lý tốn kém, các doanh nghiệp có thể khai thác tài nguyên tính toán, lưu trữ và dịch vụ phần mềm trực tiếp trên nền tảng đám mây với mức độ linh hoạt và khả năng mở rộng gần như vô hạn.

Oracle Cloud Infrastructure (OCI) là một trong những nền tảng đám mây đang phát triển mạnh, nổi bật với thế mạnh về cơ sở dữ liệu, khả năng xử lý hiệu năng cao, đồng thời cung cấp hệ sinh thái dịch vụ đa dạng từ IaaS, PaaS đến SaaS. Ngoài ra, OCI còn chú trọng vào bảo mật, tự động hóa và hỗ trợ trí tuệ nhân tạo, đáp ứng tốt nhu cầu chuyển đổi số toàn diện.

Đối với Việt Nam, khi chương trình chuyển đổi số quốc gia đang được thúc đẩy mạnh mẽ, việc nghiên cứu và nắm bắt công nghệ Oracle Cloud là cần thiết để sinh viên, nhà nghiên cứu và doanh nghiệp có thể tiếp cận, làm chủ công nghệ, từ đó ứng dụng vào thực tế. Đây chính là lý do đề tài “Tìm hiểu và khai thác dịch vụ đám mây của Oracle Cloud” được lựa chọn.

\textbf{2. Mục tiêu nghiên cứu}

Trình bày tổng quan về điện toán đám mây và kiến trúc của Oracle Cloud Infrastructure.

Làm quen với các dịch vụ cơ bản như tính toán (Compute), lưu trữ (Object Storage, Block Storage), cơ sở dữ liệu (Autonomous Database, MySQL Database).

Tiến hành thử nghiệm triển khai, cấu hình và quản lý tài nguyên trên OCI.

Phân tích ưu điểm, hạn chế và đề xuất các hướng ứng dụng của OCI tại Việt Nam.

Hình thành kỹ năng tư duy và thực hành giải pháp trên môi trường đám mây.

\textbf{3. Phạm vi nghiên cứu}

Các dịch vụ nền tảng: IaaS (máy chủ ảo, lưu trữ, mạng), PaaS (cơ sở dữ liệu, phân tích dữ liệu), SaaS (ứng dụng doanh nghiệp).

Các công cụ triển khai và quản lý: OCI Console, Command Line Interface (CLI), Terraform.


Đối tượng nghiên cứu chủ yếu là sinh viên, nhà nghiên cứu và doanh nghiệp vừa và nhỏ có nhu cầu ứng dụng điện toán đám mây.

\textbf{4. Phương pháp nghiên cứu}

Nghiên cứu tài liệu: Thu thập và tổng hợp kiến thức từ tài liệu chính thức của Oracle, giáo trình, bài báo khoa học và các nguồn tham khảo đáng tin cậy.

Thực hành trực tiếp: Tạo tài khoản OCI, triển khai thử nghiệm máy chủ ảo, lưu trữ dữ liệu, thiết lập cơ sở dữ liệu và sử dụng một số dịch vụ AI/ML.

Đánh giá so sánh: Xem xét các yếu tố hiệu suất, chi phí, độ tin cậy và khả năng mở rộng của OCI, so với nhu cầu thực tế của các tổ chức.

\textbf{5. Bố cục đề tài}

Ngoài phần mở đầu và kết luận, bài tiểu luận được xây dựng có bố cục như sau:

\textbf{Chương 1: }Tổng quan về điện toán đám mây.

\textbf{Chương 2: }Công nghệ ảo hóa của Oracle Cloud.

\textbf{Chương 3: }Tìm hiểu và khai thác dịch vụ điện toán đám mây Oracle Cloud.
