\plainsection{KẾT LUẬN}
Qua quá trình nghiên cứu và thực hành, có thể khẳng định rằng Oracle Cloud Infrastructure (OCI) là một nền tảng điện toán đám mây toàn diện, mang lại nhiều lợi ích cho doanh nghiệp và tổ chức. OCI cung cấp các dịch vụ đa dạng, từ hạ tầng ảo hóa, cơ sở dữ liệu, đến các dịch vụ trí tuệ nhân tạo, giúp triển khai ứng dụng, quản lý dữ liệu và tối ưu hóa quy trình một cách hiệu quả. Hạ tầng mạnh mẽ, khả năng mở rộng linh hoạt cùng hiệu năng cao là những điểm nổi bật, đáp ứng nhu cầu về tính sẵn sàng, bảo mật và hiệu quả hoạt động.

Quá trình thực hành và triển khai các dịch vụ trên OCI cho thấy nền tảng này không chỉ mạnh về hạ tầng mà còn thân thiện với người dùng. Việc quản lý tài nguyên, cấu hình mạng, triển khai máy chủ và sử dụng các công cụ giám sát, bảo mật giúp đảm bảo an toàn dữ liệu, giảm thiểu rủi ro và tăng cường kiểm soát cho tổ chức. Điều này minh chứng rằng OCI là một môi trường đáng tin cậy, hỗ trợ các doanh nghiệp khai thác tối đa tiềm năng công nghệ đám mây.

Đặc biệt, các dịch vụ thông minh của OCI, như OCI Language Service, mở ra khả năng phân tích ngôn ngữ, nhận diện thực thể và phân tích cảm xúc, hỗ trợ đa ngôn ngữ và xử lý dữ liệu một cách thông minh. Những tính năng này chứng tỏ Oracle Cloud không chỉ là nền tảng hạ tầng mà còn là công cụ phát triển ứng dụng thông minh, phù hợp với xu hướng chuyển đổi số và trí tuệ nhân tạo hiện nay.

Tổng kết lại, mặc dù vẫn còn một số hạn chế về chi phí và mạng lưới trung tâm dữ liệu, nhưng OCI với những ưu điểm về hiệu suất, bảo mật, tính mở rộng và khả năng ứng dụng thông minh vẫn là lựa chọn tối ưu cho các tổ chức muốn triển khai giải pháp đám mây hiện đại, an toàn và hiệu quả. Tiểu luận này đã cung cấp cái nhìn toàn diện, từ lý thuyết đến thực hành và ứng dụng, giúp nâng cao hiểu biết và khả năng khai thác các dịch vụ đám mây của Oracle.
